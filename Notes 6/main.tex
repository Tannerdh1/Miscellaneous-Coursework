\documentclass{article}
\usepackage[utf8]{inputenc}

\title{Notes 6}
\author{Tanner Hammond, CPSC 361}
\date{November 2021}

\begin{document}

\maketitle

Class Notes 10/25:\\
clusterssh/cssh - the ability to ssh into a group of hosts. Also provided a window with the ability to type into all consoles at once. \\
First had to install openssh-server \\
form is cssh computer1 computer2 \\
/etc/hosts  192.168.134 sean to give the ip the alias sean for ease.
Can also define clusters in /etc/cluster with a tag and a list of hosts. \\
example from what we were doing in class: group1 sean cameron. \\


Class Notes 10/27:\\
Fully qualified domain name (FQDN): example http://cpsc.roanoke.edu/fall2021/course.html \\
Port number: example http://cpsc.roanoke.edu:80/ \\

http talks on port 80 by default \\
https - 443 \\
ssh - 22 \\

goals: Document root: /Web, port 80 \\
/PrivateWeb - Document root /PrivateWeb, port 1088 - requires username and password \\

http://192.168.0.131/~username/A.html should go to that user's home directory and show that file (if there). \\

sudo apt-get update && sudo apt-get upgrade \\
sudo apt-get install apache2 \\

/etc/apache2 - apache2.conf, ports.conf, sites-available/enabled.conf \\
Got /Web and /PrivateWeb setup during this class, but was still working on the passwords. \\
I created the directories /PrivateWeb and /Web. In apache2.conf I added Directory parts for /PrivateWeb and /Web. In ports.conf I added listen 1088. I changed the conf file in sites-available for port 80 to have the document root /Web. I created another conf file for port 1088 and the document root /PrivateWeb. I then made it accessible using a2ensite conf-file-name.conf and did sudo service apache2 restart. \\
Class Notes 11/1:\\
Got the password to work. In the apache2.conf in the directory section, add to it : \\
AuthType Basic
AuthName "Restricted"
AuthUserFile /var/www/.htpasswd
Require valid-user

To create the htpasswd file, use sudo htpasswd -c /directory-you-want-to-store-the-htpasswd-files/.htpasswd username. You will then be asked to input the password. The -c is to create the file.\\

For the ~username goal from the last class, we need to use the module mod\_userdir. We also need to create users to test this on, so I created 3 users each with a couple of html files in there to be able to test. \\
You have to go to mods-available and find userdir.conf and enable it using a2enmod userdir. \\
Class Notes 11/3:\\
Setting up the NIS. sudo apt-get install nis. Set the domain name to tannernis. \\
Before doing anything with the server and clients, had to make sure I had 3 users with home directories in /tannerhome and that they had passwords. \\
sudo make -f ../makefile all in /var/yp/tannernis. \\
sudo service nis start. \\
Sean was the server, Davis and I were clients. As the clients, we had to have our service stopped and wait for Sean to get everything up and the turn it on.\\
Server has nis service started, clients have ypserv. \\
\end{document}
