%%%%%%%%%%%%%%%%%%%%%%%%%%%%%%%%%%%%%%%%%%%%%%%%%%%%%%%%%%%%%%%%%%%%%%%%%%%%%%%%%%%%
%Do not alter this block of commands.  If you're proficient at LaTeX, you may include additional packages, create macros, etc. immediately below this block of commands, but make sure to NOT alter the header, margin, and comment settings here. 
\documentclass[12pt]{article}
\usepackage{amsmath,amsthm,amssymb,amsfonts, enumitem, fancyhdr, color, comment, graphicx, environ, algorithm, algpseudocode}




%\pagestyle{fancy}
%\setlength{\headheight}{65pt}
\newenvironment{problem}[2][Problem]{\begin{trivlist}
\item[\hskip \labelsep {\bfseries #1}\hskip \labelsep {\bfseries #2.}]}{\end{trivlist}}
\newenvironment{sol}
    {\emph{Solution:}
    }
    {
    \qed
    }
\specialcomment{com}{ \color{blue} \textbf{Comment:} }{\color{black}} %for instructor comments while grading
\NewEnviron{probscore}{\marginpar{ \color{blue} \tiny Problem Score: \BODY \color{black} }}
%%%%%%%%%%%%%%%%%%%%%%%%%%%%%%%%%%%%%%%%%%%%%%%%%%%%%%%%%%%%%%%%%%%%%%%%%%%%%%%%%



%%%%%%%%%%%%%%%%%%%%%%%%%%%%%%%%%%%%%%
%Do not alter this block.
\begin{document}
%%%%%%%%%%%%%%%%%%%%%%%%%%%%%%%%%%%%%%

\title{Homework \#5 \\ CPSC 250 \\ Due: Friday, October 11th \\ Tanner Hammond}%replace X with the appropriate number
\date{}

\maketitle

For problems 1-2, your answers doesn't have to be in latex. You can CLEARLY write/draw it on 
the printout. You can adjust the white space by changing the number in the vspace following each
question.

\begin{enumerate}
\item (5) Using Figure 10.1 (included at the end of this file.) as a model, illustrate the 
result of each operation in the sequence
PUSH($S$, 4), PUSH($S$, 3), PUSH($S$, 2), POP($S$), POP($S$), PUSH($S$, 5) and PUSH($S$, 6) on 
an initially empty stack S stored in array $S[1 \cdots 5]$. \vspace{50px}
\newline
$[4]$ \newline
$[4,2]$ \newline
$[4]$ Returns 2 \newline
$[] $Returns 4 \newline
$[5]$\newline
$[6]$

\item (5) Using Figure 10.2 (included at the end of this file.) as a model, illustrate the 
result of each operation in the sequence ENQUEUE(Q,1), ENQUEUE(Q,2), ENQUEUE(Q,3), DEQUEUE(Q)
DEQUEUE(Q), and ENQUEUE(Q,2) on an initially empty queue Q stored in
array $[1 \cdots 6]$. \vspace{50px}\newline
$[1]$ \newline
$[1,2]$ \newline
$[1,2,3]$ \newline
$[2,3]$ Returns 1 \newline
$[3]$ Returns 2 \newline
$[3,2]$

\item (5) Each loop iteration of LIST-SEARCH procedure requires two test: one for $x \neq L.nil$
and one for $x.key \neq k$. Rewrite LIST-SEARCH to eliminate the test for $x \neq L.nil$ in
each iteration. Recall this is a doubly linked list with a sentinel which has fields
$prev, next, \textrm{ and } key$. Hint: the sentinel is the key.

\begin{algorithm}[H]
\begin{algorithmic}
\Procedure{LIST-SEARCH}{$L, k$}
  \State $x = L.nil.next$
  \State $L.nil.key = k$
  \While{$x.key \neq k$}
   \State $x = x.next$
  \EndWhile
  \State\Return $x$
\EndProcedure
\end{algorithmic}
\end{algorithm}

\item (10) Write an algorithm that takes a list of parenthesis ([, ], {, }, (, ))
and returns true if and only if the parenthesis are balanced. A sequence of parenthesis is 
balanced if every opened parenthesis is matched by its equivalent closing parenthesis. For 
example ( isn't balanced, since the opened parenthesis isn't closed. (] isn't balanced since
( isn't closed by its corresponding closed parenthesis. ([]){}()() is balanced since each opened 
parenthesis is closed by its corresponding closing parenthesis. Your algorithm should return 
true when given an empty list. It should also be $O(n)$ and use a stack. 

\begin{algorithm}[H]
\begin{algorithmic}
\Procedure{IS-BALANCE}{$A$}
 \State $S$ \Comment $S$ is a stack.\newline
 for x in A do \newline
    \indent $if(x=='(') or (x=='{') or (x=='[') then $\newline
    \indent \indent $s.push(x)$ \newline
    \indent $else$ \newline
        \indent \indent if S.pop()  \neq  $)$ or S.pop() \neq  $]$ or S.pop \neq  $}$ then \newline
          \indent \indent \indent  $return false$ \newline
        \indent \indent$end if$ \newline
    \indent $end if$ \newline
$end for$ \newline
$if S.size() == 0 then$ \newline
   \indent $return true$ \newline
$end if$ \newline
\EndProcedure
\end{algorithmic}
\end{algorithm}

\item (30) For each of the four types of lists in the following table, what is the asymptotic 
worst-case running time for each dynamic-set operation listed.
\begin{center}
Search: O(n), O(n), O(n), O(n) \newline
Insert: O(1), O(1), O(1), O(1)\newline
Delete: O(1), O(1), O(1), O(1) \newline
Successor: O(n), O(1), O(n), O(1) \newline
Predecessor: O(n), O(1), O(n), O(1) \newline
Minimum: O(n), O(1), O(n), O(1) \newline
Maximum: O(n), O(n), O(n), O(n) \newline
\end{center}

\item (10)
\begin{enumerate}
\item Use the grade-school multiplication algorithm to multiply 10821 by 11409.\newline
Answer:123,456,789
\item What is the time complexity of the grade-school multiplication algorithm to multiply two $
n$-digit numbers? Justify your answer.
\newline
Answer:$\Theta(n^2)$
\newline
Justification: It's a constant time because it isn't iterating through any structure. It's just a constant function that is done in one line using two ints.
\item What is the maximum number of digits required by the result of multiplying two $n$-digit 
numbers?\newline
Answer: 2n
\end{enumerate}



\end{enumerate}
%Copy the following block of text for each problem in the assignment.
%\begin{problem}{1}
%\end{problem}
%%%%%%%%%%%%%%%%%%%%%%%%%%%%%%%%%%%%%%%%
%Do not alter anything below this line.
\end{document}