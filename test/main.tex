\documentclass[12pt]{article}
\setlength{\topmargin}{-.8in}  
\setlength{\oddsidemargin}{-.1in} % 0
\setlength{\evensidemargin}{-.1in} % 0
\setlength{\textwidth}{6.7in} % 6.5
\setlength{\textheight}{9.5in}



\title{CPS 130 Homework 8 - Solutions}
\date{}

\begin{document}
\maketitle
\thispagestyle{empty}

\vspace{-2cm} 

\begin{enumerate}

\item (CLRS 9.3-3) Show how {\sc Quicksort} can be made to run in $O(n \log
  n)$ time in the worst case.

  {\bf Solution:} Modify the {\sc Partition} procedure of {\sc Quicksort}
  to use the {\sc Select} algorithm to choose the median of the input array
  as the pivot element.  The worst-case running time of {\sc Select} is
  linear, so we do not increase the time requirement of {\sc Partition},
  and selecting the median as pivot guarantees the input is split into two
  equal parts, so that we always have the best-case partitioning.  The
  running time of the entire computation is then given by the recurrence
  $T(n) = 2\,T(n/2) + O(n) = O(n\lg n)$.

\item (CLRS 9.3-5)  Suppose that you have a ``black-box'' worst-case
  linear-time median subroutine. Give a simple, linear-time algorithm that
  solves the selection problem for an arbitrary order statistic. 

  {\bf Solution:} Let $A[1 \dots n]$ denote the given array and denote the
  order statistic by $k$.  The black-box subroutine on $A$ returns the
  $(n/2)$ element.  If $k=n/2$ then we are done.  Else, we scan through $A$
  and divide into two groups $A_1,A_2$ those elements less than $A[n/2]$
  and those greater than $A[n/2]$, respectively.  If $k < n/2$, we find the
  order statistic for the $k^{th}$ element in $A_1$.  If $k > n/2$, we find
  the order statistic for the $(n/2 - k)^{th}$ element in $A_2$.  An
  example algorithm is as follows: 
 
  \begin{center}
  \framebox{\begin{minipage}{3.1in}
  {\tt SELECTION($A,k$)}
  \begin{itemize}
  \item[]{\tt BLACK-BOX($A$)}

  {\tt IF $k=n/2$ return $A[n/2]$}

  {\tt DIVIDE($A$) /* returns $A_1,A_2$ */}

  {\tt IF $k < n/2$ SELECTION($A_1,k$)}

  {\tt ELSE SELECTION($A_2,n/2-k$)}
  \end{itemize}
  \texttt{END SELECTION}
  \end{minipage}}
  \end{center}

The cost of computing the median using the black-box subroutine is $O(n)$,
and the cost of dividing the array is $O(n)$.  Let $T(n)$ be the cost of
computing the $k-th$ order statistic using the algorithm described above.
Then
%
\begin{eqnarray*}
  T(n) & \leq & cn + T(n/2)   \\
       &   =  & c(n + n/2 + n/4 + n/8 + \dots + T(1))   \\  
       & \leq & 2 cn\\
       &   =  & O(n)
\end{eqnarray*}

\item Let $A$ be a list of $n$ (not necessarily distinct) integers.
  Describe an $O(n)$-algorithm to test whether any item occurs more than
  $\lceil n/2 \rceil$ times in $A$.

  {\bf Solution:} If an element occurs more than $\lceil n/2 \rceil$ times
  in a $A$ then it must be the median of $A$. However, the reverse is not
  true, so once the median is found, you must check to see how many times
  it occurs in $A$. The algorithm takes $O(n)$ time provided you use linear
  selection and $O(n)$ space.

  \begin{center}
  \framebox{
    \begin{minipage}{4.75in}
    {\sc Test}($A,n$) {\tt
    \begin{enumerate}
    \item[1] Use linear selection to find the median $m$ of $A$.
    \item[2] Do one more pass through $A$ and count the number of occurences
      of $m$.
      \begin{itemize}
        \item[-] if $m$ occurs more than $\lceil n/2 \rceil$ times then return
        YES; 
        \item[-] otherwise return NO.
      \end{itemize}
    \end{enumerate} }
  \end{minipage}}
  \end{center}

\item (CLRS 9.3-7) Describe an $O(n)$ algorithm that, given a set $S$ of
  $n$ distinct numbers and a positive integer $k \le n$, determines the $k$
  numbers in $S$ that are closest to the median of $S$. 

  {\bf Solution:} Assume for simplicity that $n$ is odd and $k$ is even.
  If the set $S$ was in sorted order, the median is in position $n/2$ and
  the $k$ numbers in $S$ that closest to the median are in positions
  $(n-k)/2$ through $(n+k)/2$.  We first use linear time selection to find
  the $(n-k)/2$, $n/2$, and $(n+k)/2$ elements and then pass through the
  set $S$ to find the numbers less than $(n+k)/2$ element, greater than the
  $(n-k)/2$ element, and not equal to the $n/2$ element.  The algorithm
  takes $O(n)$ time as we use linear time selection exactly three times and
  traverse the $n$ numbers in $S$ once.


\end{enumerate}


\end{document}