\documentclass{article}
\usepackage[utf8]{inputenc}

\title{HW#7}
\author{Tanner Hammond}
\date{October 2020}

\begin{document}

\maketitle

\newline The database to store data about students can be done with several tables. Some of the relations in the table will be in Boyce-Codd Normal Form. A relvar is in Boyce-Codd Normal Form if and only if every left-irreducible FD's determinant is a candidate key. The first and main table would be Students table with the attributes:
\newline Student Number, Gender, Advisors, Majors, SAT Scores, High School Rank, Degree Date, Degree, Address, Social Organisations, Cumulative GPA, and Courses.
\newline The Courses table holds the attributes:
\newline Student Number, Semester Year, Semester Season, Course, Grade, Units, and Instructor.
\newline The candidate keys are Student Number and $\{$Student Number, Semester Year, Semester Season$\}$. The foreign key is Student number due to it referring to the primary key in Students to link the two tables together.
\newline Irreducible functional dependencies:
\begin{eqnarray*}
      Student Number &\rightarrow& Gender \\
      Student Number &\rightarrow& Gender\\
      Student Number &\rightarrow& Gender\\
      Student Number &\rightarrow& Advisors\\
      Student Number &\rightarrow& Majors\\
      Student Number &\rightarrow& SAT Scores\\
      Student Number &\rightarrow& High School Rank\\
      Student Number &\rightarrow& Degree Date\\
      Student Number &\rightarrow& Degree\\
      Student Number &\rightarrow& Address\\
      Student Number &\rightarrow& Social Organisations\\
      Student Number &\rightarrow& Cumulative GPA\\
      Student Number &\rightarrow& Courses\\
      Student Number, Semester Season, Semester Year &\rightarrow& Course Name\\
      Student Number, Semester Season, Semester Year &\rightarrow& Grade\\
      Student Number, Semester Season, Semester Year &\rightarrow& Units\\
      Student Number, Semester Season, Semester Year &\rightarrow& Instructor
     \end{eqnarray*}
     
\end{document}


Step 1: Write Relvar R with all of the attributes:
Relvar R:
