% --------------------------------------------------------------
% This is all preamble stuff that you don't have to worry about.
% Head down to where it says "Start here"
% --------------------------------------------------------------
 
\documentclass[12pt]{article}
 
\usepackage[margin=1in]{geometry} 
\usepackage{amsmath,amsthm,amssymb}
\usepackage{algorithmic}
\newcommand{\N}{\mathbb{N}}
\newcommand{\Z}{\mathbb{Z}}
 
\newenvironment{theorem}[2][Theorem]{\begin{trivlist}
\item[\hskip \labelsep {\bfseries #1}\hskip \labelsep {\bfseries #2.}]}{\end{trivlist}}
\newenvironment{lemma}[2][Lemma]{\begin{trivlist}
\item[\hskip \labelsep {\bfseries #1}\hskip \labelsep {\bfseries #2.}]}{\end{trivlist}}
\newenvironment{exercise}[2][Exercise]{\begin{trivlist}
\item[\hskip \labelsep {\bfseries #1}\hskip \labelsep {\bfseries #2.}]}{\end{trivlist}}
\newenvironment{reflection}[2][Reflection]{\begin{trivlist}
\item[\hskip \labelsep {\bfseries #1}\hskip \labelsep {\bfseries #2.}]}{\end{trivlist}}
\newenvironment{proposition}[2][Proposition]{\begin{trivlist}
\item[\hskip \labelsep {\bfseries #1}\hskip \labelsep {\bfseries #2.}]}{\end{trivlist}}
\newenvironment{corollary}[2][Corollary]{\begin{trivlist}                      
\item[\hskip \labelsep {\bfseries #1}\hskip \labelsep {\bfseries #2.}]}{\end{trivlist}}
\newenvironment{definition}[2][definition]{\begin{trivlist}                      
\item[\hskip \labelsep {\bfseries #1}\hskip \labelsep {\bfseries #2.}]}{\end{trivlist}}
 
\begin{document}
 
% --------------------------------------------------------------
%                         Start here
% --------------------------------------------------------------
 
%\renewcommand{\qedsymbol}{\filledbox}
 
\title{Homework \#3}%replace X with the appropriate number
\author{Tanner Hammond\\ %replace with your name
CPSC 395 - Analysis of Algorithms
\\ Due: Monday, 21} %if necessary, replace with your course title
\date{}
\maketitle

\begin{enumerate}
\item Exercise 5.1-2 (I want the algorithm, not a description) \\
Describe an implementation of the procedure RANDOM(a,b) that only makes calls
to RANDOM(0,1). What is the expected running time of your procedure, as a
function of a and b? \\
A good way to think about this is thinking about it as a binary representation due to the calls are only 0 and 1. \\
\\
RANDOM(a,b) 
\begin{algorithmic}
\IF{a == b}
\STATE return a
\ENDIF
\STATE Let n = $\ceil{log_2(b-a+1)}$ (Part of the binary representation approach, n is the number of digits in binary so it will be the amount of times the for loop will go) 
\STATE range = b-a
\STATE result = 0 
\FOR{i=1 to n (Will call Random(0,1) n times)}
\STATE rand = RANDOM(0,1) (sets rand to the random number)
\STATE result = result + r (adds r to the result)
\ENDFOR
\IF{result $>$ range (If result is bigger than the range, it will continue the process over)} 
\STATE return RANDOM(a,b)
\ELSE
\STATE  return a + result (returns the result + a since it will be a valid number)
\ENDIF
\end{algorithmic}

\\
\item Exercise 5.2-3
Use indicator random variables to compute the expected value of the sum of n dice. \\
Let random variable $D_i$ be the value on the ith dice. The values of $D_i$ are from the set \{1,2,3,4,5,6\}. They have the same probability $1/6$ due to it being a 6-sided dice. So Pr($D_i = k$).
The expectation of a single dice $D_i$ is \\
E$[D_i]$ = $\sum_{k=1}^{6} kPr(D_i=k)$ \\
= (1+2+3+4+5+6)/6 \\
= 3.5 \\
For multiple dice, it will be 3.5 times the number of dice. So for n dice, it will be 3.5n.

\item Exercise 5.3-2 \\
Professor Kelp decides to write a procedure that produces at random any permutation
besides the identity permutation. He proposes the following procedure:
PERMUTE-WITHOUT-IDENTITY(A)\\
1 n = A.length\\
2 for i = 1 to n - 1 \\
3 swap A[i] with A[RANDOM(i+1,n)] \\
Does this code do what Professor Kelp intends? \\
\\
So for this code to be what he intends, it must give permutations that is not the identity permutation. There are n! possible permutations of n given numbers. The identity permutation doesn't change the order of the given numbers. Let's use an example to see the permutations produced. Let's consider the algorithm for n = 3, A[1,2,3]. To produce all permutations except the identity permutation, it must produce 3!-1 permutations since n! permutations are possible including identity. So it would have to produce 5 permutations. The for loop has two iterations, i = 1 to 3-1=2. The first iteration takes the values 2 and 3 so only two permutations are possible. The second iteration, the value is only 3 so only one permutation is possible. Since that is also a permutation of the first iteration, a total of only 2 permutations are possible instead of 5. So it doesn't give the identity permutation but fails to produce the other permutations. So it doesn't do as intended.

\item Exercise 5.4-1 (Briefly describe how you got your answer. Assume a year has 365 days)
How many people must there be in a room before the probability that someone has the same birthday as you do is at least 1/2? How many people must there be before the probability that at least two people have a birthday on July 4 is greater than 1/2? \\

First we have to find the least number of people required for the probability of someone having the same birthday is at least 1/2. So, Probability P where none of the n people has the same birthday $<$ 0.5. P = (364/365)$^n$ due to 365 days in the year and only one day which is the similar birthday and to the n power for the amount of people. nln(364/365) $<$ ln(0.5). n $>$ (ln(0.5)/ln(364/365)). This becomes n = 253. So there must be 253 other in the room to have the same birthday as you. For the probability of two people having the same birthday on July 4th problem, let X be the number of people that their birthday is July 4th. X is a random variable  with the parameter p = 1/365. P(X=k) = $(^n_k)p^k(1-p)^{n-k}$ where n is the number of the people in the room. We must find P(X=0) + P(X = 1) $<$ 0.5 so we the probability of having 2 or more is at least 1/2. We must calculate with P(X=0) + P(X = 1) and the previous math from the first part of the question. So P(X=0) + P(X=1) = (364/365)$^n$ + n(364/365)$^{n-1}$ 1/365 = (364/365)$^{n-1}$(364+n/365). The n that satisfies is 613 and is proved when plugged into the previous, (364/365)$^{n-1}$(364+n/365).

\end{enumerate}
 
% --------------------------------------------------------------
%     You don't have to mess with anything below this line.
% --------------------------------------------------------------
 
\end{document}