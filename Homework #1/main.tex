\documentclass{article}
\usepackage[utf8]{inputenc}

\title{Homework 1}
\author{CPSC463\\
        Tanner Hammond}
\date{September 2021}

\begin{document}

\maketitle

4. What arguments can you make against the idea of a single language for
all programming domains?\\
The main domains are scientific application, business application, artificial intelligence, and web software. For a single language to be used for all of them it will need to be large and complex. It will be difficult to learn and maintain. Most users would be learning different subsets and ignore a lot of the other features. With how large and complex it would be, compilation and execution would be expensive and inefficient. \\

6. What common programming language statement, in your opinion, is
most detrimental to readability? \\
Overloading operators can be hard on readability. Operators can have their meanings completely changed, so it can a lot of confusion if not mentioned or written in any documentation. This confusion would make things difficult for readers or people trying to do debug or improve the code.\\

9. Explain the different aspects of the cost of a programming language. \\
The total cost of a programming language has a lot of factors. There's the cost of training programmers to use the language. The simplicity and orthogonality and experience of the programmers will determine how the training goes. The larger and more complex languages are going to more difficult and take longer to learn. There's the cost of writing programs which is dependent on the writability. Both the cost of training and writing programs can also be influenced by the programming environment. The cost of executing programs is impacted by the language's design. If a language does a lot of run-time type checks, it will hurt the execution time. Another one is the cost of poor reliability. If there is low reliability, then critical failures could happen. Then there is the cost of maintaining programs, which can include making corrections and adding new functionality. Maintenance is primarily dependent on readability since a lot of people besides the author will do maintenance. \\

10. What are the arguments for writing efficient programs even though
hardware is relatively inexpensive? \\
Making programs efficient is good practice. If you are able to a more efficient version of a program, then there are resources being wasted that don't need to be. Efficient programs will have better execution time, will have better reliability and maintenance. Inefficient code will be slower, but can also give errors and require a lot of maintenance and time.\\

16. Write an evaluation of some programming language you know, using the
criteria described in this chapter.\\
Java has good reliability due to their strong type-checking and exception handling. I think it had good writability. You can do many things in multiple different ways, of course this could hurt the readability if people aren't as familiar with this and if they don't know if they do different things. Java's automatic garbage collection can be nice too. I think it overall has pretty easy readability as well, even with their feature multiplicity.

\end{document}
