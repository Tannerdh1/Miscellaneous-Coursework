%%%%%%%%%%%%%%%%%%%%%%%%%%%%%%%%%%%%%%%%%%%%%%%%%%%%%%%%%%%%%%%%%%%%%%%%%%%%%%%%%%%%
%Do not alter this block of commands.  If you're proficient at LaTeX, you may include additional packages, create macros, etc. immediately below this block of commands, but make sure to NOT alter the header, margin, and comment settings here. 
\documentclass[12pt]{article}
\usepackage{amsmath,amsthm,amssymb,amsfonts, enumitem, fancyhdr, color, comment, graphicx, environ, algorithm, algpseudocode}




%\pagestyle{fancy}
%\setlength{\headheight}{65pt}
\newenvironment{problem}[2][Problem]{\begin{trivlist}
\item[\hskip \labelsep {\bfseries #1}\hskip \labelsep {\bfseries #2.}]}{\end{trivlist}}
\newenvironment{sol}
    {\emph{Solution:}
    }
    {
    \qed
    }
\specialcomment{com}{ \color{blue} \textbf{Comment:} }{\color{black}} %for instructor comments while grading
\NewEnviron{probscore}{\marginpar{ \color{blue} \tiny Problem Score: \BODY \color{black} }}
%%%%%%%%%%%%%%%%%%%%%%%%%%%%%%%%%%%%%%%%%%%%%%%%%%%%%%%%%%%%%%%%%%%%%%%%%%%%%%%%%


%%%%%%%%%%%%%%%%%%%%%%%%%%%%%%%%%%%%%%
%Do not alter this block.
\begin{document}
%%%%%%%%%%%%%%%%%%%%%%%%%%%%%%%%%%%%%%

\title{Homework \#7 \\ CPSC 250 \\ Due: Wednesday, November 6th \\ Tanner Hammond}%replace X with the appropriate number
\date{}

\maketitle

For a random variable $X$ we indicate that $X$ takes on the value $v$ with probability $p$ by writing Pr$(X = v) = p$.

\begin{enumerate}
\item (10)
Indicate whether or not each of the following are random variables. If any isn't a random variable explain why.
\begin{enumerate}
 \item $X = \begin{cases}
       -1, &\textrm{with probability }1/2 \\
       0, &\textrm{with probability }1/4 \\
       2, &\textrm{with probability }1/8 \\
       4, &\textrm{with probability }1/16
       \end{cases}$\newline
It isn't a random variable because the probability doesn't add up to 1.       
       
       
 \item $X = \begin{cases}
       0, &\textrm{with probability }1
       \end{cases}$ \newline
It is a random variable because the probability adds up to 1. Even if it only has one outcome.
       
 \item $X = \begin{cases}
       1, &\textrm{with probability }2/15 \\
       2, &\textrm{with probability }0 \\
       3, &\textrm{with probability }1/15 \\
       4, &\textrm{with probability }2/15 \\
       5, &\textrm{with probability }0
       \end{cases}$ \newline
It isn't a random variable because the probability doesn't add up to 1.
 \item For integer $i \in [0, \infty)$ Pr$(X = i) = 1/2^i$ \newline
It is a random variable because the probability equals 1.
 
\end{enumerate}

\item (10)
Compute the expected value of the following random variables.
\begin{enumerate}
 \item $X = \begin{cases}
       0, &\textrm{with probability }1/2 \\
       1, &\textrm{with probability }1/4 \\
       2, &\textrm{with probability }1/8 \\
       4, &\textrm{with probability }1/8
       \end{cases}$\newline
0 * 1/2 + 1 * 1/4 + 2 * 1/8 + 4 * 1/8 = 0 + 1/4 + 1/4 + 1/2 = 1
 \item $X = \begin{cases}
       -2, &\textrm{with probability }1/5 \\
       -1, &\textrm{with probability }1/5 \\
       0, &\textrm{with probability }1/5 \\
       1, &\textrm{with probability }1/5 \\
       2, &\textrm{with probability }1/5
       \end{cases}$\newline
-2 * 1/5 + -1 * 1/5 + 0 * 1/5 + 1 * 1/5 + 2 * 1/5 = -2/5 - 1/5 + 1/5 + 2/5 = 0
 \item $X = \begin{cases}
       1, &\textrm{with probability }2/5 \\
       8, &\textrm{with probability }1/5 \\
       27, &\textrm{with probability }1/5 \\
       64, &\textrm{with probability }1/10 \\
       125, &\textrm{with probability }1/10
       \end{cases}$\newline
1 * 2/5 + 8 * 1/5 27 * 1/5 + 64 * 1/10 + 125 * 1/10 = 2/5 + 8/5 + 27/5 + 32/5 + 125/10 = 26.3
\end{enumerate}

\item (10)
In each of the following, compute the expected value of $X$.
\begin{enumerate}
 \item $X$ takes on integer values in the $[1,n]$. Pr$(X = i) = 1/n$. Compute the expectation of
 $X$. $\Sigma_{x = 1}^{n} 1/n$ =  1/1 + 1/2 + 1/3 + 1/4 + ..... = 2
 \item $X$ takes on integer values in the range $[0, \infty)$. Pr$(X = i) = 1/2^{i+1}$. $\Sigma_{n=0}^{\infty} 1/2^{i+1}$ = $\Sigma_{n=0}^{\infty} 1/2^{1}$ + $\Sigma_{n=1}^{\infty} 1/2^{i}$ = $1/2 + 1/2^2 + 1/2^3....$ = 1
\end{enumerate}

\item (15)
Let $X$ and $Y$ be random variable defined as follows.
 $$X = \begin{cases}
 0, &\textrm{with probability }1/2 \\
 1, &\textrm{with probability }1/4 \\
 2, &\textrm{with probability }1/4 \\
 \end{cases}$$
 
 $$Y = \begin{cases}
 0, &\textrm{with probability }1/3 \\
 2, &\textrm{with probability }1/3 \\
 3, &\textrm{with probability }1/3 \\
 \end{cases}$$
 
 Use $X$ and $Y$ to compute the following random variable.
 \begin{enumerate}
 \item $X+Y$ = $0 * 1/2 + 1 * 1/4 + 2 * 1/4 + 0 * 1/3 + 2 * 1/3 + 3 * 1/3$ = 1/4 + 1/2 + 2/3 + 1 = 29/12
 \item $X - 3/4$ 0 * 1/2 + 1 * 1/4 + 2 * 1/4 = 1/4 + 2/4 = 3/4 - 3/4 = 0
 \item $XY$ (3/4)(2*1/3 + 3 * 1/3) = (3/4)(2/3 + 1) = (3/4)(5/3) = 5/4
 \end{enumerate}

\item (10)
The following code uses a random number generator RAND(). It returns 0 with probability $9/10$ and 1 with probability $1/10$. 
  
 \begin{algorithm}
 \begin{algorithmic}[1]
 \Procedure{UntilISucceed}{}
 \State count $\gets$ 1;
 \While{$\textrm{RAND()} \neq 1$}
  \State count $\gets$ count + 1
 \EndWhile
 \State \Return count
 \EndProcedure
\end{algorithmic}
\end{algorithm}
\begin{enumerate}
 \item What are the possible values returned by UntilISucceed()? $(1,\infty)$
 \item What is the probability that UntilISucceed() returns 0? 9/10
 \item What is the probability that UntilISucceed() returns 1? 1/10
 \item Let the random variable $X$ denote the output of UntilISucceed(). What type of random 
 variable is $X$? Continous
 \item What is the expected value of $X$? (You can apply a theorem instead of computing it by 
 hand.) $\Sigma_{k=1}^n$ Pr(X=k)
\end{enumerate}

\item (10) Consider the following game. A random number 
generator guesses your birthday (Ignore leap years), if the random number generator guesses 
wrong you get a dollar, otherwise, you get paid \$365? Let $X$ be a random variable that counts 
the amount of money you get after playing this game (negative values represent you paying money 
and positive values represent you being paid).
\begin{enumerate}
\item What is the expected value of $X$? 364
\item A game is called fair if the expected payout is 0. What value should the amount paid for a 
wrong guess be changed to in order to make this a fair game? You should 365 dollars if it guesses correct instead of getting paid that money. It's a very unfair game otherwise, the player only gets paid and doesn't pay anything.
\end{enumerate}

\item (10)

Suppose that you want to output 0 with probability 1/2 and 1 with probability 1/2.
At your disposal is a procedure BIASED-RANDOM, that outputs either 0 or 1. It
outputs 1 with some probability $p$ and 0 with probability $1-p$, where $0 < p < 1/2$,
but you do not know what $p$ is. Give an algorithm that uses BIASED-RANDOM
as a subroutine, and returns an unbiased answer, returning 0 with probability 1/2
and 1 with probability 1/2. (Hint: consider the probabilities of events when BIASED-RANDOM is
called twice. Some events should be desirable. If at first you don't succeed, try again.)

\begin{algorithm}[H]
\begin{algorithmic}
\Procedure{UNBIASED}{()}
\State a= BIASED-RANDOM();
\State b = BIASED-RANDOM();
\State if a \textless  b then return 0;
\State if a \textgreater  b then return 1;
\EndProcedure
\end{algorithmic}
\end{algorithm}

Express the expected running time of your algorithm as a function of $p$ in big-theta notation.

\item (15) Let the random variable $X$ be the value returned by a uniform random generator that generates integers in the interval $[1,n]$? Uniform means that each outcome has equal probability.
\begin{enumerate}
\item What is the expectation of $X$? 1/n
\item What is the expectation of $2X - n$? 1/n
\item What is the expectation of $X^3$? 1/n
\end{enumerate}
\end{enumerate}

\end{document}